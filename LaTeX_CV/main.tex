\documentclass[a4paper]{article}
    \usepackage{fullpage}
    \usepackage{amsmath}
    \usepackage{amssymb}
    \usepackage{textcomp}
    \usepackage[utf8]{inputenc}
    \usepackage[T1]{fontenc}
    \usepackage{hyperref}
    \usepackage{geometry}
    \usepackage[most]{tcolorbox}
    \usepackage{xcolor} % for color definitions
    \geometry{left=0.8in,right=0.8in,bottom=0.8in,top=0.8in}

    % DEFINITIONS FOR RESUME %%%%%%%%%%%%%%%%%%%%%%%

\newcommand{\area} [2] {
    \vspace*{-9pt}
    \begin{verse}
        \textbf{#1}   #2
    \end{verse}
}

\newcommand{\lineunder} {
    \vspace*{-8pt} \\
    \hspace*{-18pt} \hrulefill \\
}

\newcommand{\header} [1] {
    {\hspace*{-18pt}\vspace*{6pt} \textsc{#1}}
    \vspace*{-6pt} \lineunder
}

\newcommand{\employer} [3] {
    { \textbf{#1} (#2)\\ \underline{\textbf{\emph{#3}}}\\  }
}

\newcommand{\contact} [3] {
    \vspace*{-10pt}
    \begin{center}
        {\Huge \scshape {#1}}\\
        #2 \\ #3
    \end{center}
    \vspace*{-8pt}
}

\newenvironment{achievements}{
    \begin{list}
        {$\bullet$}{\topsep 0pt \itemsep -2pt}}{\vspace*{4pt}
    \end{list}
}

\newcommand{\schoolwithcourses} [4] {
    \textbf{#1} #2 $\bullet$ #3\\
    #4 \\
    \vspace*{5pt}
}

\newcommand{\school} [4] {
    \textbf{#1} #2 $\bullet$ #3\\
    #4 \\
}

\tcbset{
  myprojectbox/.style={
    colback=gray!10,
    colframe=gray!40,
    boxrule=0.3pt,
    arc=3pt,
    left=6pt,
    right=6pt,
    top=4pt,
    bottom=4pt
  },
  techpill/.style={
    colback=blue!10,
    colframe=blue!50,
    boxrule=0.2pt,
    arc=3pt,
    boxsep=1pt,
    left=3pt,
    right=3pt,
    top=1pt,
    bottom=1pt,
    nobeforeafter
  }
}

% END RESUME DEFINITIONS %%%%%%%%%%%%%%%%%%%%%%%

    \begin{document}
\vspace*{-40pt}

%==== Profile ====%
\vspace*{-10pt}
\begin{center}
	{\Huge \scshape {Joshua Pascal Lambertz}}\\
	Wien $\cdot$ \href{mailto:joshuapascallambertz@gmail.com}{joshuapascallambertz@gmail.com} $\cdot$ 1578 8767226 $\cdot$ \href{https://github.com/BatuhanHerdem}{github.com/JoshuaLambertz}\\
\end{center}

\begin{center}
\begin{minipage}{0.93\textwidth}
\centering
\small
Entwickler mit praktischer Erfahrung in der Software- und Webentwicklung, spezialisiert auf CPP und Python. Ich arbeite gerne im Team an abwechslungsreichen Projekten und interessiere mich besonders für Game Development und Computer Vision.
\end{minipage}
\end{center}

%==== Education ====%
\header{Ausbildung}
\textbf{Berliner Hochschule für Technik}\hfill Berlin\\
B.Sc. Architektur \hfill 04.2021 - 09.2024

\vspace{3mm}
\noindent\textbf{Universität Wien}\hfill Wien\\
B.Sc. Informatik \hfill 10.2025 - aktuell
\vspace{8mm}

%==== Experience ====%
\header{Berufserfahrung}
\vspace{1mm}

\textbf{iteratec GmbH} \hfill Düsseldorf\\
\textit{Werkstudent – Softwareentwicklung} \hfill 04.24 – 07.25\\
\vspace{-5mm}
\begin{itemize} \itemsep -1pt
    \item Entwicklung kundenspezifischer Softwarelösungen
    \item Umsetzung interner Tools und Anwendungen
    \item Arbeit mit Java, Azure, AWS, Automatisierung, Webanwendungen, OOP und Datenbanken
\end{itemize}
\vspace{3mm}
\textbf{mindsquare AG} \hfill Bielefeld\\
\textit{Werkstudent – Webentwicklung} \hfill 08.23 – 03.24\\
\vspace{-5mm}
\begin{itemize} \itemsep -1pt
    \item Umsetzung von Anforderungen anhand zugewiesener Tickets
    \item Sicherstellung der Einhaltung von Qualitätsstandards und Unternehmensrichtlinien
    \item Weiterentwicklung und Optimierung bestehender Systeme auf Basis neuer Anforderungen
    \item Arbeit mit PHP, SQL, WordPress und HTML
\end{itemize}
\vspace{3mm}
\textbf{Heinrich-Heine-Universität} \hfill Düsseldorf\\
\textit{Tutor für Theoretische Informatik} \hfill 04.23 – 07.23\\
\vspace{-5mm}
\begin{itemize} \itemsep -1pt
    \item Durchführung und Bewertung von Übungsleistungen im Bereich Theoretische Informatik
\end{itemize}
\vspace{8mm}


\header{Fähigkeiten}
\begin{tcolorbox}[myprojectbox]
\vspace{2mm}
\begin{tabular}{ l l }
    \textbf{Programmiersprachen:} & Java, Python, C++, JavaScript, PHP\\[6pt]
    \textbf{Webtechnologien:} & Bootstrap, JSON, SQL, CSS, HTML \\[6pt]
    \textbf{Frameworks und Tools:} & React, Node.js, Git, Linux \\[6pt]
    \textbf{Cloud / DevOps:} & AWS, GitHub Actions \\[6pt]
\end{tabular}
\end{tcolorbox}


\vspace{8mm}


\header{Projekte}

\begin{tcolorbox}[myprojectbox]
\textbf{Automatisierte Generierung und Korrektur von CYK/CNF-Aufgaben} \hfill \textit{Bachelorarbeit} \\[4pt]
\tcbox[techpill]{\small\textit{Python, CI/CD, Algorithmendesign, GitHub Actions, GitHub Classroom}} \\[4pt]
Entwicklung eines automatisierten Systems zur Generierung und Korrektur von Aufgaben zu CYK- und CNF-Algorithmen.\\
Integriert in GitHub Classroom zur automatisierten Bewertung studentischer Lösungen mit kontinuierlicher Rückmeldung.\\
Das System wird aktuell aktiv in der Lehre an der Heinrich-Heine-Universität Düsseldorf eingesetzt.
\end{tcolorbox}
\vspace{4pt}

\begin{tcolorbox}[myprojectbox]
\textbf{IK Portal} \hfill \textit{iteratec GmbH} \\[4pt]
\tcbox[techpill]{\small\textit{Azure, Java, Spring, Node.js}} \\[4pt]
Webbasiertes Kundenprojekt zur sicheren Verwaltung und zum Austausch von Dokumenten.\\
Vollständig cloudbasiert und ohne Datenbank umgesetzt.
\end{tcolorbox}
\vspace{4pt}

\begin{tcolorbox}[myprojectbox]
\textbf{eCMR System} \hfill \textit{iteratec GmbH} \\[4pt]
\tcbox[techpill]{\small\textit{AWS, Java, Spring, Automatisierung, Docker, MySQL, Computer Vision}} \\[4pt]
Entwicklung eines automatisierten Systems zur Umwandlung von CMRs (Frachtbriefen) in digitale Versionen (eCMR).\\
Das System nutzt AWS-Services und eine automatisierte Umwandlung von CMR-Fotos in digitale eCMRs.
\end{tcolorbox}
\vspace{4pt}


\begin{tcolorbox}[myprojectbox]
\textbf{Splitter} \hfill \textit{\href{https://github.com/BatuhanHerdem/Splitter}{GitHub}} \\[4pt]
\tcbox[techpill]{\small\textit{Java, Spring, Docker, PostgreSQL, GitHub OAuth, HTML}} \\[4pt]
Webanwendung zur Verwaltung gemeinsamer Ausgaben in Gruppen.\\
Eigenständig im Team entwickelt im Rahmen eines Universitätsprojekts.
\end{tcolorbox}
\vspace{10pt}



\header{Sprachen}
\textbf{Deutsch} \hspace{18mm} \textbf{Englisch} \hspace{18mm} \textbf{Spanisch}\hspace{18mm}\\
Muttersprache \hspace{10.15mm} Fließend \hspace{19mm} Grundkenntnisse\\

\end{document}